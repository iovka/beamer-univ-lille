% on peut choisir entre aspectratio 43 et 169
\documentclass[10pt,aspectratio=169]{beamer}
\usepackage{beamer-univ-lille}

\usepackage{fontspec}
\usepackage[francais]{babel}
\usepackage[T1]{fontenc}
\usepackage[utf8]{inputenc}

% la police d'après la charte graphique
\setmainfont{Verdana}


% petite macro pour afficher la table des matières; n'a rien à voir avec le style
\newcommand{\TOC}{\begin{frame}{Plan}
\tableofcontents
\end{frame}}
\newcommand{\TOCcs}{
\begin{frame}{Plan}
\tableofcontents[currentsection]
\end{frame}}

% -----------------------------------------------------
\author[Collectif]{Première Version:\\Iovka Boneva,\\Toute Contribution\\Est Bienvenue}
%\date{2017--2018}
\title[Beamer Univ Lille]{Style Beamer aux couleurs de Université de Lille}
% --------------------------------------------------------------------



\begin{document}

% Page de titre
{
\titlepagestyle
\begin{frame}
  \begin{columns}[c]
    \column{0.3\textwidth}
    \column{0.7\textwidth} \maketitle    
  \end{columns}
\end{frame}
}

\section{Introduction}
\TOC
\TOCcs

\begin{frame}{Que propose ce style ?}
  \begin{itemize}
  \item définition des couleurs de la palette Univ Lille d'après \\
    \url{https://identite.intranet.univ-lille.fr/règles-dutilisation/couleurs}
  \item définition des couleurs pour  quelques éléments standard de Beamer
  \end{itemize}
\end{frame}

\begin{frame}{Ce que ce style n'est pas}
  Ce style \alert{n'est pas un gabarit officiel} fourni par l'Université de Lille.

  \bigskip
  Il est fait par des personnels de l'université, donné tel quel à être utilisé par ceux qui le jugeraient utile
\end{frame}



\section{Aperçu des éléments du style}
\TOCcs

\begin{frame}{Éléments beamer}
  Ces couleurs sont utilisées par défaut, n'hésitez pas à faire votre propre cuisine dans le fichier de style !
  \begin{block}{Les blocks}
    Le titre d'un block utilise une des couleurs de la palette, modifiable dans le fichier de style.
  \end{block}
  \begin{exampleblock}{Les exemples}
    Le titre des blocks d'exemple utilise une des couleurs de la palette
  \end{exampleblock}
  \begin{alertblock}{Block en alerte}
    Texte sur lequel on veut attirer l'attention
  \end{alertblock}
  Texte \alert{en alerte}.
\end{frame}

\begin{frame}{Comment l'utiliser}
  \begin{itemize}
  \item importer les définitions avec \texttt{\textbackslash{}usepackage\{beamer-univ-lille\}} %\texttt{\bs{}usepackage{latex-cours-univ-lille}}
  \item utiliser ce fichier d'exemple comme trame.
  \item ne pas oublier \texttt{\textbackslash{}setmainfont\{Verdana\}} dans le préambule pour utiliser la police recommandée par la charte graphique de l'université. Peut-être il vous faudra installer la police
  \end{itemize}
\end{frame}

\section{Comment contribuer}
\TOCcs

\begin{frame}{Toute contribution est bienvenue}
  Si quelqu'un se sent l'envie de rendre cette première ébauche moins artisanale:
  \begin{itemize}
  \item Ce serait bien d'avoir un \alert{thème couleur} fait dans les règles de l'art, à utiliser avec \texttt{\textbackslash{}usecolortheme\{univLille\}}
  \item Si vous améliorez le style, partagez votre contribution !
  \item Pour cela, faire un pull request sur github, ou envoyer un mail à \texttt{iovka.boneva@univ-lille.fr}
  \item N'oubliez pas de changer ce fichier d'exemple pour qu'il reflète votre contribution
  \end{itemize}
\end{frame}


% Dernière page
{
  \titlepagestyle
  \begin{frame}{}
    \begin{center}
      Enjoy !
    \end{center}
  \end{frame}
}


\end{document}
